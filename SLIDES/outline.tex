% Created 2015-03-29 Sun 16:05
\documentclass[bigger]{beamer}
\usepackage[utf8]{inputenc}
\usepackage[T1]{fontenc}
\usepackage{fixltx2e}
\usepackage{graphicx}
\usepackage{longtable}
\usepackage{float}
\usepackage{wrapfig}
\usepackage{rotating}
\usepackage[normalem]{ulem}
\usepackage{amsmath}
\usepackage{textcomp}
\usepackage{marvosym}
\usepackage{wasysym}
\usepackage{amssymb}
\usepackage{hyperref}
\tolerance=1000
\usepackage{listings} \lstset{  language={C++},  basicstyle=\ttfamily\tiny}
\mode<beamer>{\usetheme{Madrid}}
\usetheme{default}
\author{Sandeep Koranne}
\date{27 March 2015}
\title{Concepts of Programming Languages}
\hypersetup{
  pdfkeywords={},
  pdfsubject={},
  pdfcreator={Emacs 24.4.1 (Org mode 8.2.10)}}
\begin{document}

\maketitle
\begin{frame}{Outline}
\tableofcontents
\end{frame}


\begin{frame}[label=sec-1]{Instructor Background}
\begin{itemize}
\item Chief Scientist at Mentor Graphics
\item Research background in algorithms, data structures, parallel programming, compiler optimization and graph theory
\item Programming background in Common Lisp, Fortran, C, C++, Python, Erlang, etc
\end{itemize}
\end{frame}

\begin{frame}[label=sec-2]{Previous Research}
\begin{columns}
\begin{column}{0.5\textwidth}
\includegraphics[width=.9\linewidth]{../../PICTURES/pcci_book.png}
\end{column}
\end{columns}
\begin{block}{Handbook of Open Source Tools}
\includegraphics[width=.9\linewidth]{../../PICTURES/hbook_oss.png}
\end{block}
\end{frame}

\begin{frame}[label=sec-3]{Programming Experience}
\begin{columns}
\begin{column}{0.3\textwidth}
\includegraphics[width=.9\linewidth]{../../PICTURES/soc_time.jpg}
\includegraphics[width=.9\linewidth]{../../PICTURES/mips_8_crop.png}
\includegraphics[width=.9\linewidth]{../../PICTURES/gl3d.jpg}
\end{column}
\begin{column}{0.3\textwidth}
\includegraphics[width=.9\linewidth]{../../PICTURES/4cube.jpg}
\includegraphics[width=.9\linewidth]{../../PICTURES/drc.jpg}
\includegraphics[width=.9\linewidth]{../../PICTURES/stiener_tree.jpg}
\end{column}
\end{columns}
\end{frame}

\begin{frame}[label=sec-4]{Course Outline}
\begin{block}{Motivation}
\end{block}
\end{frame}


\begin{frame}[label=sec-5]{Introduction}
\begin{block}{A simple slide}
This slide consists of some text with a number of bullet points:

\begin{itemize}
\item the first, very @important@, point!
\item the previous point shows the use of the special markup which
translates to the Beamer specific \emph{alert} command for highlighting
text
\end{itemize}
\end{block}
\end{frame}
\begin{frame}[label=sec-6]{Quadratic Equations}
\begin{block}{Basic form}
This is the equation $x=\frac{a}{b}$
\end{block}
\end{frame}

\begin{frame}[label=sec-7]{Lisp code verbatim}
\begin{example}[Example code]
This is an example
(defun fac(n) (+ n 2))
\end{example}
\end{frame}

\begin{frame}[label=sec-8]{C++ Code Example}
\begin{block}{Fibonacci Numbers}
\lstinputlisting[language=c++]{../EXAMPLE_CODE/fibo.cpp}
(END) 
\end{block}
\end{frame}
% Emacs 24.4.1 (Org mode 8.2.10)
\end{document}
