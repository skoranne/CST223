% Created 2015-04-01 Wed 20:18
\documentclass[bigger]{beamer}
\usepackage[utf8]{inputenc}
\usepackage[T1]{fontenc}
\usepackage{fixltx2e}
\usepackage{graphicx}
\usepackage{longtable}
\usepackage{float}
\usepackage{wrapfig}
\usepackage{rotating}
\usepackage[normalem]{ulem}
\usepackage{amsmath}
\usepackage{textcomp}
\usepackage{marvosym}
\usepackage{wasysym}
\usepackage{amssymb}
\usepackage{hyperref}
\tolerance=1000
\usepackage{listings} \lstset{  language={C++},  basicstyle=\ttfamily\tiny}
\mode<beamer>{\usetheme{Madrid}}
\usetheme{default}
\author{Sandeep Koranne}
\date{\today}
\title{Taxonomy of Programming Languages}
\hypersetup{
  pdfkeywords={},
  pdfsubject={},
  pdfcreator={Emacs 24.4.1 (Org mode 8.2.10)}}
\begin{document}

\maketitle

\begin{frame}[label=sec-1]{Motivation}
\pause
\begin{block}{Critical view of language features}
\end{block}
\end{frame}

\begin{frame}[label=sec-2]{Classification of Programming Languages}
\begin{block}{Style}
\begin{itemize}[<+->]
\item Imperative (Procedural)
\item Functional
\item Logic
\item Object Oriented
\item Declarative
\end{itemize}
\end{block}
\end{frame}

\begin{frame}[label=sec-3]{Evolution of Programming Languages}
\begin{itemize}[<+->]
\item The original (circa 1955) concern was efficiency
\item consider
\end{itemize}
\begin{eqnarray}
y & = & 2*(y+5y^2) + \sin(y)\cos(y) \\
dx & = & \frac{x-y}{step}\\
dy & = & \sin(y)dx
\end{eqnarray} 
\begin{itemize}
\item It was widely believed that no computer program would ever be able to optimize these sort of programs
\item Today, writing such code even in Fortran and C++ is considered low level
\item Only in the most critical cases is assembly language programming used.
\end{itemize}
\end{frame}

\begin{frame}[label=sec-4]{Language features and their contrast}
\begin{itemize}[<+->]
\item There is no such thing as a \emph{perfect} language
\item However, certain language features have become idiomatic
\item Such as the use of abstract data types in C++ for templates
\item Use of asynchronous messaging in Erlang
\item The use of map-reduce (from Common Lisp)
\item Array and slice operations from Fortran
\item List comprehensions from Common Lisp
\item Garbage collection (from Lisp and then Java)
\item Use of packages (Python)
\item Strength of awk, sed, grep, bash, Tcl,
\item Use of SQL for declarative database programming
\end{itemize}
\end{frame}

\begin{frame}[label=sec-5]{The goal of this course is to give an introduction to the above}
\end{frame}

\begin{frame}[label=sec-6]{Outline of this course}
\begin{center}
\begin{tabular}{llll}
Lecture & Content & Lab & HW\\
4/2 & Chap 1-2 & C language & HW1\\
4/9 & Chap 3-4 & C language & HW2\\
4/16 & Chap 5-6 & Fortran & HW3\\
4/23 & Chap 7-8 & Common Lisp & HW4\\
4/30 & Chap 9-10 & C++ & HW5\\
5/7 & Chap 11 & C++ & HW6\\
5/14 & Chap 12 & C++ & HW7\\
5/16 & Chap 13 & Erlang & HW8\\
5/21 & No class & - & -\\
5/28 & Chap 14 & Python & HW9\\
6/4 & Chap 15 & Various & HW10\\
6/11 & Chap 16 & Various & Final\\
 &  &  & \\
\end{tabular}
\end{center}

\begin{block}{Each class may have a in-class quiz}
\end{block}
\end{frame}
% Emacs 24.4.1 (Org mode 8.2.10)
\end{document}
